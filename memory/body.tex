% 1) Introduction
\chapter{Introducción}

Existe un interés por parte de la comunidad investigadora de tener libre acceso a las métricas correspondientes con su trabajo, como puede verse con las varias iniciativas y soluciones existentes para lidiar con este problema, tales como Google Scholar, Scopus, o Web of Science entre otras.

Nuestro objetivo es acabar con el agobio producido por la disponibilidad de tantos estándares proponiendo otro nuevo estándar~\cite{xkcd_standards}, mediante el desarrollo de una solución \emph{software} que cualquier investigador pueda usar de manera libre .

% 2) Goals
\chapter{Objetivos}

Para completar el desarrollo de nuestra solución, existen una serie de objetivos que se deben completar en orden:

\begin{enumerate}
  \item Identificar las fuentes de datos
  \item Descarga de todos los datos disponibles
  \item Limpieza y análisis de los datos
  \item Interfaz gráfica que muestre los datos
\end{enumerate}

% 3) Prior art
\chapter{Estado del Arte}

Existen muchas fuentes que tratan de ser completas y tener toda la información disponible.

(Mencionar Scholar, Scopus, Web of Science, etc.)

Otro intento es la aplicación propietaria Publish or Perish~\cite{publish_or_perish}, que aunque hace un buen trabajo ofreciendo un único punto donde acceder a la información, se queda corto a la hora de agregarla y filtrarla.

% 4) Methodology
\chapter{Metodología}

Con el propósito de investigar qué fuentes hay disponibles, se ha seguido una metodología "explorativa", realizando pruebas con las APIs correspondientes a cada sitio (si la ofrecen).

% 5) Implementation and development
\chapter{Implementación y desarrollo}

Las tecnologías utilizadas son Svelte~\cite{svelte} para la interfaz gráfica junto con Python~\cite{python} para el servidor.

(Explicar por qué)

% 6) Results
\chapter{Resultados}

El desarrollo se ha pausado por motivos mayores :-)

% 7) Conclusions
\chapter{Conclusiones y trabajos futuros}

Como futuro trabajo, podrían incluirse incluso más fuentes, o rediseñar la interfaz para que usarla sea más eficaz.
