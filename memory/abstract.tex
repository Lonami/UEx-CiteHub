\chapter*{Resumen}
\thispagestyle{empty}

El trabajo aquí presentado aborda los problemas con los que se encuentran los investigadores de cualquier campo cuando necesitan comprobar, de manera manual, una o varias fuentes para determinar cómo de bien rinden sus publicaciones.

Entre estos problemas se encuentra la laboriosa búsqueda manual de sus propios datos o la falta de métricas personalizadas que puedan resultar útiles al investigador (yendo más allá de los comunes números de citas, \emph{h-index} o \emph{i10-index}).

Para solventarlos, se ha realizado un estudio de las soluciones existentes y propuesto el desarrollo de una aplicación que sea capaz de agregar todas las fuentes posibles de manera automática, aliviando así la carga de trabajo en el investigador interesado. Se presenta un posible diseño y por últmo se discuten las principales conclusiones y posibles vías futuras de trabajo.

\textbf{Palabras clave}: agregador, h-index, i10-index, citaciones.

\thispagestyle{empty}

\chapter*{Abstract}

The work presented here takes on the problems researches from any field face when they need to check, manually, one or many sources to determine how well their publications are performing.

Among these problems we may find the time-consuming manual search of their own work or the lack of custom metrics that may result helpful for the own researcher (going beyond the common number of citations, \emph{h-index} or \emph{i10-index}).

To solve these issues, we have studied the existing solutions and proposed the development of an application able to aggregate all available sources automatically, thus alleviating the workload on the interested researcher. We present a possible design and lastly we discuss the main conclusions and possible future work.

\textbf{Keywords}: aggregator, h-index, i10-index, citations.

\thispagestyle{empty}
